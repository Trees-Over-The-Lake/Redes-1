\section{Unidade 1 - Introdução}

\subsection{Quais são as funções das sete camadas do modelo de
\newline referência OSI?}

\subsubsection{Aplicação}

Suporte a aplicações de rede (e.g., FTP, SMTP, HTTP).

\subsubsection{Apresentação}

Permite que as aplicações interpretem significado de
dados, por exemplo, criptografia, compactação, convenções específicas
da máquina.

\subsubsection{Sessão}

Sincronização, verificação, recuperação de troca de dados.

\subsubsection{Transporte}

Transferência de dados processo-processo.

\subsubsection{Rede}

Roteamento de pacotes da origem ao destino.

\subsubsection{Enlace}

Transferência de dados entre elementos vizinhos da rede.

\subsubsection{Física}

Transmissão de bits e meios de transmissão.

\subsection{Qual a diferença de visibilidade entre as camadas de rede e
enlace?}

A camada de rede é focada em controlar os protocolos
da rede, enquanto a camada de enlace fica por conta 
de controle de fluxo e controle de rede. Dessa forma,
a camada de rede tem a visibilidade apenas dos protocolos
e a camada de enlace dos bits para encontrar erros.

\subsection{Tanto a camada de rede quanto a de transporte, são
responsáveis pela transferência de dados, qual a diferença
entre elas?}

A diferença entre as duas camadas está no elemento em que 
são focados, enquanto a camada de transporte tem uma visão
mais abstrata e fica por conta do tipo de protocolo escolhido
- como os protocolos \emph{UDP} e \emph{TCP} - a camada de rede
fica por conta de controlar o IP que será adicionado ao pacote:
\emph{IP} e \emph{PORTA}.
 
\subsection{O que significa Broadcasting na camada de rede e na de
enlace?}

\emph{Broadcasting} é o processo de enviar uma mesma mensagem
para vários dispositivos conectados na rede simultaneamente.