\section{Unidade 3 - Camadas de Transporte e Aplicação}

\subsection{Faça um paralelo entre a porta para a camada de transporte e o endereço
de rede para a camada de rede.}

A porta é a função principal da camada de transporte que define a porta de origem e de destino, 
também define o tipo do pacote se é UDP ou TCP. A principal função da camada de rede é adicionar 
o IP de origem e o de destino.

\subsection{Por que a camada de transporte dá sentido à pilha de protocolos?}

A ideia é que pode existir protocolos se sobrepondo, é possível ter um protocolo UDP
sobre um protocolo SMTP, por exemplo, a camada de transporte apenas padroniza para um protocolo
que a origem e o destino irão conseguir comunicar. Após chegar no destino, a aplicação pode
decidir como usar do outro procolo que foi colocado por cima do UDP.

\subsection{O significa uma aplicação do tipo one shot? Os protocolos TCP pode ser
utilizado na camada de transporte para esse tipo de aplicação? E o UDP?
Justifique suas respostas.}

Aplicações \emph{One-Shot} são aplicações que enviam uma mensagem sem se preocuparem
se essa mensagem vai conseguir chegar ou se materá a integridade até no destino.
Essas aplicações são normalmente para uso de aplicativos que usam áudio ou vídeo em tempo-real,
perder um pacote ou outro não faz grande diferença para a qualidade da mensagem final.
Isso é uma especificidade do UDP, não sendo possível fazer com o TCP. O TCP necessariamente
exige uma conexão e checagem se o pacote chegou corretamente, caso o pacote não chegue corretamente
será enviado outro logo após. 

\subsection{Explique como executamos as primitivas de soquetes para o TCP em uma
aplicação do tipo cliente-servidor}

\begin{tabular}{|c|c|}
    \hline
    Primitiva & Significado \\
    \hline
    SOCKET & Criar um novo ponto final de comunicação \\ 
    BIND & Anexar um endereço local a um soquete \\
    LISTEN & Anunciar a disposição para receber conexões; mostrar o tamanho da fila \\
    ACCEPT & Bloquear o responsável pela chamada até uma tentativa de conexão recebida \\
    CONNECT & Tentar estabelecer uma conexão ativamente \\
    SEND & Enviar alguns dados através da conexão \\
    RECEIVE & Receber alguns dados da conexão \\
    CLOSE & Encerrar a conexão \\
    \hline
\end{tabular}

\subsubsection*{Servidor}

SOCKET $\rightarrow$ Especifica o formato do endereço, tipo de serviço (e.g., fluxo de
bytes confiável) e protocolo.

BIND $\rightarrow$ Atribui um endereço de rede ao soquete. \footnote{Observação: a primitiva 
socket não atribui diretamente um endereço porque alguns processos utilizam endereços
anteriormente definidos}

LISTEN $\rightarrow$ Aloca espaço para a fila de
chamadas recebidas.

ACCEPT $\rightarrow$ Bloqueia o responsável pela chamada até que o servidor
receba uma tentativa de conexão. Quando o servidor recebe uma conexão, ele pode desviá-la para
outro processo (thread) e voltar à espera por uma nova conexão.

\subsubsection*{Cliente}

SOCKET $\rightarrow$ Como no servidor, esta primitiva especifica o formato do endereço,
tipo de serviço (e.g., fluxo de bytes confiável) e protocolo. \footnote{A primitiva BIND não é necessária
no cliente, pois seu endereço é irrelevante para o servidor.}

CONNECT $\rightarrow$ Tenta estabelecer uma conexão
com o servidor.

SEND / RECEIVE $\rightarrow$ As partes devem ter uma política para
definir de quem é a vez de enviar dados.

CLOSE $\rightarrow$ \emph{Desconexão simétrica:} cada parte faz sua desconexão
para informar que não enviará mais dados.


\subsection{Como os campos Sequence e Acknowledgement numbers atuam na
transferência confiável de dados do TCP}

O \emph{Sequence Number} identifica qual a posição do pacote completo
essa fragmentação que chegou representa o \emph{Acknowledgement Number}
representa qual será o próximo byte esperado do próximo fragmento. 

\subsection{Explique como os campos SYN, ACK e FIN funcionam no estabelecimento e
no término de conexões TCP}

\subsubsection*{SYN}

Campo feito para iniciar o processo de conexão entre a origem e o destino.

\subsubsection*{ACK}

Campo para representar se o \emph{Acknowledgement Number} é válido.

\subsubsection*{FIN}

Campo para finalizar a conexão.

\subsection{Descreva o controle de fluxo no TCP}

Existe um contador no computador de origem e no de destino da quantidade de pacotes
que estão sendo enviados para a quantidade de pacotes que estão chegando no destino.
Caso o destino esteja recebendo menos pacotes que o destino está enviando a origem irá
diminuir o número de pacotes enviados conforme o destino avisa. O contrario também acontece
se a rede estiver livre, o número de pacotes enviados será aumentado.

\subsection{Descreva o controle de congestionamento no TCP}

A fim de o emissor não esgotar a capacidade de transporte da rede, é estabelecida
um tamanho para a janela de congestionamento. No início da conexão, a janela de
congestionamento começa vazia e o emissor vai mandando uma quantidade
crescente de pacotes até a rede avisar que não consegue receber uma quantidade
maior daquele emissor. Depois disso, o emissor passa a enviar a quantidade fixa de
pacotes por tempo até que esgote 100\% a capacidade da janela. 
Caso a rede detecte novos emissores e veja um risco de
congestionamento, ela pede para todos os emissores conectados limparem a sua janela de
congestionamento e recomeçarem o processo de encontrar o tamanho máximo da nova janela.