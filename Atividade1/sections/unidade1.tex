\subsection{Em breve, teremos um terminal doméstico e seguro conectado a
Internet permitindo plebiscitos instantâneos sobre questões
importantes. Nesse caso, a política atual será eliminada. Os
aspectos positivos dessa democracia direta são óbvios, analise
alguns dos aspectos negativos.}

    Um aspecto negativo que vi a algum tempo que foi citado
    por um grande programador e \emph{youtuber} chamado 
    \hyperlink{https://www.youtube.com/c/TomScottGo}{Tom Scott}.
    Ele citou algo que me fez pensar por bastante tempo.
    O grande problema de usarmos computadores para votar é a existência
    de uma chance de uma falha ou votos falsos escalarem rapidamente\cite{WEBSITE:1}. 

    Para deixar mais claro,
    se alguém mal intensionado conseguir ter acessado ao sistema de votação
    é muito fácil que o estrago seja muito grande, como trocar os votos de milhões
    de pessoas com apenas um clique e difícil de identificá-lo se for um 
    \emph{hacker} com grande conhecimento de computação.

\subsection{O presidente da XBeer resolve trabalhar com a YBeer para produzir
uma lata de cerveja invisível (medida higiênica). O presidente pede
que o jurídico analise a questão. Esse contacta o departamento de
Engenharia. Como resultado, o engenheiro-chefe entra em contato
com seu par na YBeer para discutirem os aspectos técnicos. Em
seguida, os engenheiros enviam um relatório aos departamentos
jurídicos, que discutem os aspectos legais. Por fim, os presidentes
discutem as questões financeiras do negócio. Esse é um exemplo de
protocolo em várias camadas no sentido utilizado pelas redes de
computadores? Justifique.}

Para fazer essa comparação podemos considerar que a empresa está dividida em 
camada de aplicação, sendo esse o presidente, camada de transporte, rede e enlace, sendo esse
o jurídico, e por fim, os engenheiros sendo a camada física. Toda essa divisão é feita
sem que uma das camadas influêncie no julgamento da outra.

\subsection{Um sistema tem uma hierarquia de protocolos com n camadas. As
aplicações geram mensagens com M bytes de comprimento. Em
cada uma das camadas, é acrescentado um cabeçalho com h bytes.
Qual é a fração dos dados enviados que corresponde ao tamanho
dos cabeçalhos?}

\begin{center}
    {\huge $\frac{n * h}{n * h + M}$}
\end{center}

\subsection{Determine qual das camadas do modelo TCP/IP trata de cada uma
das tarefas a seguir:}

\subsubsection{Dividir o fluxo de bits transmitidos em quadros}

Camada de enlace.

\subsubsection{Definir a rota que será utilizada na sub-rede}

Camada de rede.

\subsection{Cite dois aspectos em que os modelos de referência OSI e TCP/IP
são similares e dois em que eles são diferentes.}

Semelhanças:
\begin{itemize}
    \item Existência da camada de aplicação.
    \item Existência da camada de enlace.
\end{itemize}

Diferenças:
\begin{itemize}
    \item O modelo OSI tem uma camada de sessão e uma camada de apresentação.
    \item o modelo TCP/IP possuí uma camada que abstrai as duas chamada aplicação.
\end{itemize}

\subsection{Diferencie os protocolos TCP e UDP}

O modelo TCP foi pensado para criar uma conexão entre dois computadores.
Com isso, é possível que o envio de pacotes com verificação de erro e se
por acaso acontecer, ele pode ser resolvido reenviando o pacote. Tanto quanto
é possível ter controle da velocidade em que os pacotes estão sendo enviados
para não sobrecarregar a rede. 

O caso do UDP é específico para aplicações
\emph{One-Shot} que se beneficiam de ser um pacote mais leve e que usa não
possuí controle para envios mais rápidos dos pacotes. Usos bons para o UDP
são quando não há grande importância na integridade do conteúdo do pacote
ou quando se quer enviar muitos pacotes rapidamente com pouco \emph{overhead}
de processamento.

\subsection{Explique os termos Latência, Largura de Banda e Taxa de Dados}

\subsubsection*{Latência}

Tempo necessário para o pacote atravessar toda a rede - da origem até o destino-.

\subsubsection*{Largura de banda}

Quantidade máxima de dados que podem trafegar em uma rede de uma só vez.

\subsubsection*{Taxa de Dados}

Quantidade efetiva de dados que estão sendo transmitidos na rede.